
\documentclass[a4paper,10pt]{article}
\newcommand{\mylatex}{\textrm{\LaTeX}}
\usepackage[left=.5in, right=.5in, top=1in, bottom=1in]{geometry}
\usepackage{enumitem}
\usepackage{titlesec}
\usepackage{hyperref}
\usepackage{parskip}

% Set section and subsection formatting
\titleformat{\section}{\large\bfseries}{}{0em}{}[\titlerule]
\titleformat{\subsection}{\bfseries}{}{0em}{}

\begin{document}

\begin{center}
    {\LARGE Jennifer Kathleen Green} \\
    \vspace{0.1cm}
    \href{mailto:jkgreen@sandia.gov}{jkgreen@sandia.gov} \\
    \href{www.linkedin.com/in/jennifer-fisher-green-15a06332}{www.linkedin.com/in/jennifer-fisher-green-15a06332} \\
    \vspace{0.5cm}
\end{center}

\section*{Objective}
Strategic and results-oriented high performance computing professional with over a decade of experience in technical, project, and team leadership, dedicated to advancing the mission of Sandia National Laboratories by effectively executing strategies that align with the laboratory mission. Seeking to leverage extensive expertise in program management, team leadership, and cross-divisional collaboration to drive operational excellence and enhance performance across the Labs. Committed to engaging customers and stakeholders, negotiating agreements, and managing resources to ensure the successful delivery of quality products and services. Eager to contribute to Sandia’s mission by optimizing organizational capabilities and workforce stewardship, ultimately supporting the Laboratories in achieving their strategic objectives.\\
\\


\section*{Experience}
\textbf{Dec 2025 - Present} \\
\textbf{Manager, R\&D S\&E Computer Science, Sandia National Laboratories, Albuquerque, New Mexico - USA} \\
Interim Manager for the HPC Development Group. The group comprises the Observation, Orchestration, and Optimization and the Heterogeneous Advanced Architecture Platforms teams. Together, these teams explore future platforms and approaches to improve application throughput on emerging technologies, integration of modern tooling for future runtimes....<-fix me!!!!!!!!!!!!!

\textbf{Oct 2023 - Present} \\
\textbf{Principal, R\&D S\&E Computer Science, Sandia National Laboratories, Albuquerque, New Mexico - USA} \\
Lead the Monitoring Team Deployment Project by ensuring the availability, reliability, and accessibility of the monitoring service infrastructure and its data analyses for the production resources at Sandia National Laboratory. Provide advanced solutions to Tier 3 \& 4 problems faced in supporting HPC systems, especially surrounding user build and runtime issues, testing, and troubleshooting. Work with the Advanced Architecture team to further their mission of assessing novel architectures for future procurements.

\textbf{July 2020 – Oct 2023} \\
\textbf{Team/Technical Leader, R\&D Scientist IV, Los Alamos National Laboratory, Los Alamos, New Mexico – USA} \\
Serve as the Technical and Team Lead for the Programming \& Runtime Environments Team of the High-Performance Computing Division. Position LANL to drive scientific software stack management, system and application performance testing for the NNSA ASC Laboratories.

\textbf{June 2015 – July 2020} \\
\textbf{Team/Technical Leader, R\&D Scientist III, Los Alamos National Laboratory, Los Alamos, New Mexico – USA} \\
Serve as the Technical and Team Lead for the Programming \& Runtime Environments Team of the High-Performance Computing Division. Oversee and guide effective installation, maintenance, support and testing of the software and environments used by scientists on production high performance computers. Lead projects aimed to ensure the quality and availability of development platforms, systems and environments. Lead the NNSA/ASC laboratories in a deployment solution for delivering scientific software via Spack in production. NNSA/ASC Common Computing Environment LANL Programming Environment Lead.

\textbf{July 2013 – June 2015} \\
\textbf{R\&D Scientist II, Los Alamos National Laboratory, Los Alamos, New Mexico – USA} \\
Ensure a rich and consistent Scientific Programming Environment is maintained across all LANL production clusters. Provide excellent customer support for software, programming and analysis tools. Host workshops, work with support staff to validate and configure specialized and custom environments in support of National Security and Institutional Computing initiatives. Contribute to performance testing pre-production/production HPC resources and results analysis processes. Contribute to software development projects for tools for automating testing, maintenance and installation tasks of the programming environment and engage other DOE HPC groups to ensure consistency in environments across multiple sites.

\textbf{April 2011 – June 2013} \\
\textbf{R\&D Scientist I, Los Alamos National Laboratory, Los Alamos, New Mexico – USA} \\
Perform and interpret results of on-demand, post-maintenance, and automated nightly testing of high-performance computing production resources, including specialized hardware, parallel and network file systems, distributed resource managers and schedulers, stack software and Linux kernel troubleshooting. Serve in a leadership role in the Open|SpeedShop development team to best serve Los Alamos code developers with performance optimization requirements.

\textbf{Oct 2010 – April 2011} \\
\textbf{Graduate Research Assistant Internship, Los Alamos National Laboratory, Los Alamos, New Mexico - USA} \\
Ensure reliability, accessibility, serviceability, and performance testing within a production environment. Utilize scripting languages to augment current test scripts; research into creating/enhancing testing tools to perform customized system health and performance testing. Short-term goals of debugging current applications to measure multiple HPC systems’ health exist in tandem with objectives for providing a versatile testing environment. Long term goal to achieve qualities of interactivity, ubiquity across available and future platforms, scalability to future forecasted HPC environments while providing Software Support team with useful diagnostic and analytical tools that measure performance and assist optimization efforts.

\textbf{June 2010 – Sept 2010} \\
\textbf{Graduate Research Assistant Internship, Los Alamos National Laboratory, Los Alamos, New Mexico - USA} \\
Assisted scientists involved in Chemical Metallurgy Resource Replacement project towards goal of Nuclear Software Quality Assurance. Tasks included applying NQA-1 2008 standards in software quality assurance procedures and collaboration on data-driven software development to track software requirements’ traceability to nuclear quality standards. Research included literature surveys in Safety Software, Software Configuration Management, Nuclear Software Quality Assurance, and Department of Energy Orders.

\textbf{Jan 2010 – May 2010} \\
\textbf{Graduate Research Assistant Internship, Kentucky NSF EPSCoR, Frankfort, Kentucky - USA} \\
Research and development of prototype Kentucky Lake Visualization software using OpenGL/C++ and C\#/SQL Server 2005. Interacted with staff of Murray State University and Kentucky Virtual Observatory and Ecological Informatics System (VOEIS) involved in the development of robust infrastructure to Hancock Biological Station, consisting of sensor networks for real-time, water quality data acquisition, enhancements to existing network infrastructure, lake/weather data collection, processing, and storage. Proposed a framework for a visualization tool that is versatile for public use, yet provides drill down capabilities to serve the purpose of water researchers, based upon a data model that is adopted by freshwater monitoring groups globally, and developed a prototype single system version to demonstrate the data flow among the GUI, database, and graphical code.

\textbf{April 2009 – Dec 2009} \\
\textbf{Graduate Research Assistant Internship, Kentucky NSF EPSCoR, Frankfort, Kentucky - USA} \\
Experimental Program to Stimulate Competitive Research established through the National Science Foundation serving to foster and enhance Kentucky’s Science \& Technology resources. Preliminary studies in MPI coding and parallel processing techniques in C on IBM HS21 blade cluster (BCX). Literature review in Parallel Processing and High Performance Computing. Special project involvement to assist Kentucky researchers in converting existing code to include MPI and to develop parallelized code for specific research initiatives.

\textbf{Aug 2008 – April 2009} \\
\textbf{Graduate Research Assistant Internship, Kentucky State University, Frankfort, Kentucky - USA} \\
Hybrid systems modeling approach to the verification, testing and design of a wireless sensor network for small to mid-sized airport runway monitoring system. Conducted study into system verification methods and logic-verification of network model, as a product of the first phase of the project.

\section*{Leadership Activities}
\textbf{2025 - Present} \\
\textbf{Monitoring Systems Representative MEXT/DOE} \\
Collaborate with sister laboratories of the NNSA and MEXT partners in ensuring a shared monitoring capability for next generation supercomputers for the US Department of Energy and Japan's Ministry of Education, Culture, Sports, Science and Technology (MEXT) Institute of Physical and Chemical Research (RIKEN) Fugaku based systems

\textbf{2025 - Present} \\
\textbf{Technical Representative for NNSA's ASC HPC Software Strategy Project} \\
Serving on the Cloud Infrastructure and Programming Environments sub-working groups for the NNSA ASC program's strategy project, I contribute my expertise and represent Sandia National Laboratories' perspective in the formation of strategic direction and planning

\textbf{Summer 2025} \\
\textbf{FY25 IC L2 Milestone \#9549 Review Committee Member} \\
Focusing on the evaluation of SPARC and Empire on El Capitan, this experience involved collaboration with program leadership and code team representatives from tri-labs and HQ, where I provided insights that guided critical decision-making processes.

\textbf{March 2025} \\
\textbf{OLCF-6 Technical Review Committee} \\
I represented DOE NNSA and Sandia on the review panel for the OLCF-6 Technical Review Committee at Oak Ridge National Laboratory. In this role, I guided discussions regarding testing, support, and system software selections, ensuring that our contributions align with national priorities and best practices

\textbf{2025 - Present} \\
\textbf{TCE El Dorado \& Tachi Deployment Projects} \\
These projects ensure that the scientific software environments supplied by HPE and that which is supplied by the El Capitan and Crossroads respective software teams are usable and synchronized in support of development activities for Sandia National Laboratories' customers. The role includes leading a small team, fostering collaborations among the tri-laboratory, supplying custom solutions for specific technical challenges, and supporting our code-team's complex HPC build and runtime environments run optimally on these advanced architecture systems.

\textbf{April 2022 – Oct 2023} \\
\textbf{Acceptance Testing Lead, Crossroads ATS-3 Alliance for Computing at Extreme Scale (ACES) Procurement} \\
Lead the effort of planning, developing, coordinating activities for, and ensuring the expert execution of the requisite tests in order to verify the Crossroads, Tycho, Rocinante, and Razorback systems adhere to the requirements as stipulated in the Crossroads Statement of Work. This is a multi-disciplinary and multi-group/laboratory collaboration that is a high-visibility and high-impact project, critical to the success of the ASC program. We’re revolutionizing the way that Acceptance Testing is being orchestrated, ensuring repeatability, provenance of configurations, source code, inputs and results analysis, facilitating transparency, accountability, and continuity of the developed testing capability.

\textbf{2018 – Oct 2023} \\
\textbf{Programming Environment Lead for LANL, Common Computing Environment} \\
Lead the LANL NNSA/ASC Programming Environments Project’s team to achieve the objectives set forth by the ASC program. The goal of the Programming Environment Project is to enhance productivity of the tri-lab application development teams, operation teams, and analysts by developing and deploying user tools and programming environments to support a variety of applications running on tri-lab HPC resources. Focus areas include improving development and support for common dependencies of performance analysis, testing, and debugging tools.

\textbf{2017 – 2018, 2019 – 2020, 2021 – 2022} \\
\textbf{Programming Environment Lead for NNSA/ASC Tri-labs, Common Computing Environment} \\
Lead the NNSA/ASC Tri-lab’s Programming Environments Project’s team to define and deliver the objectives set forth by the Advanced Simulation and Computing Program. The goal of the Programming Environment Project is to enhance productivity of the tri-lab application development teams, operation teams, and analysts by developing and deploying user tools and programming environments to support a variety of applications running on tri-lab HPC resources. Focus areas include improving development and support for common dependencies of performance analysis, testing, and debugging tools.

\textbf{2020 – Oct 2023} \\
\textbf{Technical Representative for System Assurance Working Group, CORAL2 Project} \\
Guide the development of automation for assuring technical soundness and correctness in the CORAL2 system. Develop tools and collective approaches to assuring the CORAL2 Pre-Exascale system is delivered to meet the specifications agreed upon in the procurement.

\textbf{2020 – Oct 2023} \\
\textbf{Technical Representative for Software Packaging Working Group, CORAL2 Project} \\
Ensure that LANL customers’ use-cases for the software environment delivered with CORAL2 is met by the vendor. Involvement with the vendor to ensure that their vision and ours marry to meet the customers’ needs in deploying large scale scientific simulation software to their customers built with vendor supplied software. Work with the vendor to ensure container use model is adequate and in line with what they expect.

\textbf{2019 – Oct 2023} \\
\textbf{Programming Environments Lead for the Crossroads Enabling Science Team, Crossroads} \\
Serve as the lead for the implementation and delivery of the Programming Environments for Crossroad, including gauging efficacy of the vendor’s proposed solution, addressing gaps in requirements and implementation, and working with the vendor to ensure that the programming environment is desired, usable and addresses the needs of the ACES user and support communities.

\textbf{June 2020 - August 2020} \\
\textbf{Lead for Supercomputer Institute Project 6, Integration of the ECP Proxy Apps Suite into the Pavilion Test Harness, Los Alamos National Laboratory High Performance Computing Division Supercomputer Institute} \\
Lead a team of three undergraduate student interns and two mentors to successfully integrate the ECP Proxy App Suite supplied by the Exascale Computing Project.

\textbf{2014 – Oct 2023} \\
\textbf{Team Leader of the Programming \& Runtime Environments Team} \\
Guide the technical work of staff for accomplishing exemplary support of software management, testing, licensing and training for ASC and IC production HPC systems at LANL. Responsibilities include hiring and personnel management, project management, training and mentoring, deriving work-package contents for CCE, IC and ASC for our areas of responsibility. Effective interoperation among multiple HPC functional units and teams, as well as cross-division and inter-laboratory collaborations have sprung from building a solid team, developing a world-class support model, and sharing it among our colleagues in HPC.

\textbf{2014 – Oct 2023} \\
\textbf{Mentor, National Security Education Center Student Program} \\
Mentor UGS, GRA, Post-Masters and Doctoral Students, National Security Education Center Student Program. Served as a primary and secondary mentor for more than a dozen student interns since 2011. Several success stories have resulted from positive internship experiences for the students mentored.

\textbf{April 2019 – November 2019} \\
\textbf{Subject Matter Expert, Usability Team, Crossroads Technical Advisory Team} \\
Reviewed proposals from vendors for efficacy and usability of the system as proposed. Worked among the team to identify areas where clarification was required, or where the proposed solution was inadequate for LANL’s requirements for the Crossroads procurement.

\section*{Conference Proceedings and Workshops}
\begin{itemize}
    \item Green, J. "Monitoring System Deployment Management Modernization at Sandia." Presentation to LDMSCon'25, Chicago, IL. 2025 
    \item Green, K. "HPC Characterization \& Performance Monitoring." Presentation to Cyber, Business, and Information Technology Networking Group - Sandia National Laboratories, Albuquerque, NM. 2023 
    \item Green, J. “Crossroads Programming Environment Working Group.” Presentation to the Crossroads Center of Excellence Workshop III, Los Alamos, NM. 2022. 
    \item Green, J., Agelastos, A., Stroup, K. “Crossroads: Status on Design, Deployment, Acceptance, and Operation.” Presentation to the Cray User Group Meeting 2022, Monterey, CA. 2022. 
    \item Green, J. “TCE Deployment Using Spack.” Presentation to the Virtual Advanced Simulation and Computing Principle Investigators Meeting, May, 2021. Los Alamos, NM. 2021. 
    \item Green, J. “Pavilion in Ten Minutes.” Lightning Talk Presentation to the Cray User Group Virtual Meeting 2021, Los Alamos, NM. 2021. 
    \item Green, J., Sly, N., Ogas, J., Lapid, F., Magee, D., Ferrell, P., Everson, K., Seamons, C. “Acceptance Testing the Chicoma HPE Cray EX Supercomputer.” Paper publication and Presentation to the Cray User Group Virtual Meeting 2021, Los Alamos, NM. 2021. 
    \item Green, J., Patchett, J., Ferrell, P. “DOECGF’19 Los Alamos National Laboratory Site Report - State of Production Visualization Resources at Los Alamos National Laboratory, NM.” Presentation to the Department of Energy Computer Graphics Forum, DOECGF’19, Monterey, CA. 2019. 
    \item Schulz, M., Galarowicz, J., Maghrak, D., Green, J., Montoya, D.M. “How to Analyze the Performance of Parallel Codes 101: A Case Study with Open|SpeedShop.” Tutorial Presented to The International Conference for High Performance Computing, Networking, Storage and Analysis, SC’2018. Dallas, TX. 2018. 
    \item Green, J., Morin, M. “Real-Time Systems Monitoring in HPC- LANL’s HPC Division’s Production Splunk Use Cases.” NNSA Cyber-Security Summit. Livermore, CA. 2018. 
    \item Schulz, M., Galarowicz, J., Green, J., Montoya, D.M., Maghrak, D. “How to Analyze the Performance of Parallel Codes 101: A Case Study with Open|SpeedShop.” Tutorial Presented to The International Conference for High Performance Computing, Networking, Storage and Analysis, SC’2017. Denver, CO. 2017. 
    \item Green, J., Patchett, J., Cone, G. “Bringing ParaView into Production.” Presentation to the Department of Energy Computer Graphics Forum, DOECGF’17, National Renewable Energy Laboratory (NREL), Golden, CO. 2017. 
    \item Schulz, M., Galarowicz, J., Green, J., LeGendre, M., Maghrak, D. “How to Analyze the Performance of Parallel Codes 101: A Case Study with Open|SpeedShop.” Tutorial Presented to The International Conference for High Performance Computing, Networking, Storage and Analysis, SC’2016. Salt Lake City, UT. 2016. 
    \item Green, J. “Programming \& Runtime Environments for High Performance Computing.” Presentation, LANL HPC Student Seminar Series, Los Alamos, NM. 2016. 
    \item Green, J. “HPC Test: Splunk on HPC Testing at LANL.” Invited speaker, Splunk Government Day, Online. 2015. 
    \item Green, J. “HPC Benchmarking Visualization Tool.” Session presentation to .conf2014 5th Annual Splunk Conference. Las Vegas, NV. 2014. 
    \item Schulz, M., Galarowicz, J., Green, J., LeGendre, M., Maghrak, D. “How to Analyze the Performance of Parallel Codes 101: A Case Study with Open|SpeedShop.” Tutorial Presented to The International Conference for High Performance Computing, Networking, Storage and Analysis, SC’2013. Denver, CO. 2013. 
    \item Montoya, D., Green, J. “The Current State of CBTF Tool Development and a Case for Monitoring Tools.” Petascale Tools Workshop 2013. University of Wisconsin, Madison WI. 2013. 
    \item Pedicini, G., Green, J. “SPOTlight on Testing: Stability, Performance and Operational Testing of LANL HPC Clusters.” Proceedings of the International ACM/IEEE Conference for High Performance Computing, Networking, Storage and Analysis, SC’2011. Seattle, WA. 2011. 
    \item Green, J. “Design and Development of a Cluster Testing System,” Poster Presentation by special invitation to Salishan Conference on High Speed Computing 2011, Gleneden Beach, OR. 2011. 
    \item Green, J. Data Visualization of Kentucky Lake Using Sensor Networks. Masters Thesis presented to Kentucky State University, Frankfort, KY. 2010. 
    \item Green, J., Bhattacharyya, S., and Panja, B. “Real-Time Logic Verification of a Wireless Sensor Network,” CSIE ’09: Proc. 2009 WRI World Congress on Computer Science and Information Engineering. Washington, DC, USA: IEEE Computer Society, 2009, pp. 269–273.
\end{itemize}

\section*{Conference, Committees \& Programs Participation}
\begin{itemize}
    \item Chair, Reproducibility Committee AD/AE Appendices, The International Conference for High Performance Computing, Networking, Storage and Analysis, SC'2026. Chicago, IL. 2026.
    \item Vice-Chair, Reproducibility Committee, The International Conference for High Performance Computing, Networking, Storage and Analysis, SC'2025. St. Louis, MO. 2025.
    \item Reviewer, Best Papers Selection Committee, 39th IEEE International Parallel \& Distributed Processing Symposium, IPDPS'25. Milan, Italy. 2025. 
    \item Reviewer, IEEE CiSE special issue: Transforming Science through Software: Improving while delivering 100X. Computing in Science \& Engineering. 2025. 
    \item Member, LDMSCON2025 Program Committee, LDMSCON2025: LDMS Users Group Conference 2025. Chicago, IL. 2025
    \item Member, Reproducibility Committee AD/AE Appendices, The International Conference for High Performance Computing, Networking, Storage and Analysis, SC'2022. Dallas, TX. 2022. 
    \item Member, Tutorials Committee, The International Conference for High Performance Computing, Networking, Storage and Analysis, SC'2022. Dallas, TX. 2022. 
    \item Member, Research Posters Committee, The International Conference for High Performance Computing, Networking, Storage and Analysis, SC'2022. Dallas, TX. 2022. 
    \item Member, Student Poster Selection Committee, The International Conference for High Performance Computing, Networking, Storage and Analysis, SC’2019. Denver, CO. 2019. 
    \item Member, Student Volunteer Selection Committee, The International Conference for High Performance Computing, Networking, Storage and Analysis, SC’2019. Denver, CO. 2019. 
    \item Member, Student Volunteer Selection Committee, The International Conference for High Performance Computing, Networking, Storage and Analysis, SC’2018. Dallas, TX. 2018. 
    \item Steering Committee Member, OpenHPC, 2016-2017. 
    \item Member, Student Volunteer Selection Committee, The International ACM/IEEE Conference for High Performance Computing, Networking, Storage and Analysis, SC’2017. Denver, CO. 2017. 
    \item Member, Student Poster Selection Committee. The International ACM/IEEE Conference for High Performance Computing, Networking, Storage and Analysis, SC’2015. Austin, TX. 2015. 
    \item Member, Birds of a Feather Selection Committee. The International ACM/IEEE Conference for High Performance Computing, Networking, Storage and Analysis, SC’2014. New Orleans, LA. 2014. 
    \item Participant, Broader Engagement. The International ACM/IEEE Conference for High Performance Computing, Networking, Storage and Analysis, SC’2011. Seattle, WA. 2011. 
    \item Participant, Broader Engagement. The International ACM/IEEE Conference for High Performance Computing, Networking, Storage and Analysis, SC’2009. Portland, OR. 2009.
\end{itemize}
\section*{Certifications, Awards and Endorsements}
\begin{itemize}
    \item "SNL in Crossroads L2 Milestone Presentations." Thunderbird Kudos Team Award. Sandia National Laboratories, 2024.
    \item “Trinity Installation.” Distinguished Performance Award, Los Alamos National Laboratory, 2017. 
    \item “Trinity Merge.” Los Alamos Awards Program, 2017. 
    \item “Successful Deployment and Acceptance of Trinity Supercomputer.” Defense Programs Award of Excellence for Significant Contribution to the Stockpile Stewardship Program, 2016. 
    \item “Splunk Certified Knowledge Manager.” Version 6. October 9, 2014. 
    \item “Wolf Installation.” Los Alamos Awards Program, 2014. 
    \item “Endorsement.” New Manager On-Ramp. Los Alamos National Laboratory. April 2014.
\end{itemize}
\section*{Technologies and Languages}
C, Python, Shell, CMake, GnuMake, Linux, Darwin, Windows, HPC, MPI, OpenMP, Cuda, HIP, Performance Analysis Tools, Monitoring Software, HPC Visualization, C++, Kubernetes, Gitlab CICD, Github Actions, Podman, Docker, Ansible, CFEngine, Spack, RPM, Linux Administration, ParaView, \mylatex

\section*{Education}
\textbf{Aug 2007 -- May 2010 Master of Science in Computer Science Technology (GPA 3.933)} \\
    Kentucky State University, Frankfort, Kentucky - USA \\
\textbf{Jan 2007 -- Aug 2007 Undergraduate Studies in Computer Science (GPA 3.885)} \\
    Kentucky State University, Frankfort, Kentucky - USA \\
\textbf{May 2001 -- Dec 2003 Bachelor of Arts in English (GPA 3.12)} \\
    University of Kentucky, Lexington, Kentucky - USA \\
\textbf{Jan 1998 -- May 2001 University General Studies Requirements} \\
    Lexington Community College, Lexington, Kentucky – USA \\

\end{document}
